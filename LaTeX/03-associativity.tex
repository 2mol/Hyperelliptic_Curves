\documentclass[english,11pt,a4paper]{article}
\usepackage{preamble}

\begin{document}

\section{Associativity}

\begin{theorem}
  Let $\P_i \in \J$, $\P_i = \{ \q_1, \q_2 \}$, $\P_2 = \{ \q_3, \q_4 \}$ and $\P_3 = \{ \q_5, \q_6 \}$,
  \begin{align*}
    \P_1 + \P_2 = \P_3.
  \end{align*}
  Then there is an $f \in \F^*$ that has as divisor
  \begin{align*}\label{f}\tag{$\star$}
    (f) = \q_1 + \q_2 + \q_3 + \q_4 - \q_5 - \q_6 - 2 \infty.
  \end{align*}
  \begin{proof} In five of the six cases of the addition law we have our polynomial $p(x)$ passing through the points $(x_i, y_i)$ for $i = 1,..,6$. This motivates us to look 

    \begin{enumerate}[1.]\setcounter{enumi}{-1}
      \item If $\P_2 = 0$ then \eqref{f} is equal to $(f) = \q_1 + \q_2 + 2 \infty - \q_1 - \q_2 - 2 \infty = 0$, a condition that is fulfilled by $f = 1$.

      now

      \item Remember that the $\q_i$ are all pairwise distinct and $\q_i \neq \bar \q_j$ whenever $i \neq j$. Call the coordinates 
    \end{enumerate}
  \end{proof}
\end{theorem}


\end{document}