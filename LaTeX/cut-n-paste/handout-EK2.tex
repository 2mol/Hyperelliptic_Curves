\documentclass[ngerman,11pt,a4paper,titlepage]{article}

\usepackage{graphicx}
\usepackage{bbm}
\usepackage{esvect}
\usepackage{amsmath}
%\usepackage{mathtools}
\usepackage{pifont}
\usepackage{mathabx}
\usepackage[mathcal]{euscript}
\usepackage{units}
\usepackage{wrapfig}
\usepackage[super]{cite}
\usepackage[german,ngerman]{babel}
\usepackage[utf8]{inputenc}

\usepackage{fancyhdr}

\pagestyle{fancyplain}

\newcommand{\C}{\mathbbm C}
\newcommand{\R}{\mathbbm R}
\newcommand{\Q}{\mathbbm Q}
\newcommand{\Z}{\mathbbm Z}
\newcommand{\A}{\mathcal{A}}
\newcommand{\K}{\mathcal{K}}
\newcommand{\iQ}{\mathcal{Q}}

\begin{document}
\lhead{Juri Chomé}
\rhead{\today}
\cfoot{}
\subsection*{Kapitel 4 - Northcott (Fortsetzung)}
\vspace{5mm}
\textbf{Hilfssatz 4.3:}
\\*
Für jedes $x \in K \subset L$ gilt
\begin{align*}
  H_L(x) = H_K(x)^d && \text{ mit } d = [L:K] .
\end{align*}
\vspace{-3mm}
\\*
\textbf{Beweisverlauf:}
\\*
Für Einbettungen klar, da $\sigma : K \to \C$ genau $d$ Einbettungen $\sigma_i$ liefert. Bleiben die Primideale zu überprüfen:
%\end{list}
\renewcommand{\labelenumi}{(\roman{enumi})}
\begin{enumerate}
  \item Betrachte $x\in \Z_K$ mit $x\Z_K=\wp^e\A$, $\wp \notdivides \A$ und erweitere auf $L$ mit $\wp_{L} = \prod \iQ_i^{e_i}$, $\iQ_i \in \Z_L$.
    \vspace{2mm}
    \\*
    Zeige, dass $\prod |x|_{\iQ_i}=|x|_{\wp}^d.$
  \item Zeige $|x|_{\iQ_i}=|x|_{\wp}^{\theta_i}$, $\theta_i > 0$ und daher $|x|_{\iQ_i} > 1 \Leftrightarrow |x|_{\wp} > 1$.
  \item Verallgemeinere auf alle $x \in K^*$, da $x=\nicefrac{x_1}{x_2}$, $x_1,x_2\in \Z_K$ und die Bewertung multiplikativ ist.
  \item Bemerke, dass jedes Primideal $\iQ \subset \Z_L$ eindeutig mit einem $\wp\subset \Z_K$ zusammenhängt.
\end{enumerate}
Alles zusammen ergibt, dass das Produkt über alle $\wp$ genau dem über alle $\iQ$ ist und die Behauptung folgt.
%
\vspace{8mm}
\\*
\textbf{Satz (Northcott):} Für jedes $T\in \R$ gilt
\\*
\begin{align*}
  \#\{x\in K | H_K(x) \leq T \} < \infty.
\end{align*}
\textbf{Beweisverlauf:}
\\*
Klar für $K= \Q$ wegen $H(\nicefrac{r}{s})=\max \{|r|,|s| \}.$ Betrachte nun ein beliebiges $x$ und dessen charakteristisches Polynom
\begin{align*}
  P(t)=(t-\sigma_1(x)) \cdots (t-\sigma_d(x)) \in \Q [t].
\end{align*}
Definiere $\K := \Q(\sigma_1(x),\ldots,\sigma_d(x))$ und betrachte einen Koeffizienten $q$ von $P$ als Polynom in $\Z [\sigma_1(x),\ldots,\sigma_d(x)]$.
\vspace{5mm}
\\*
Schätze nun $H$ ab mit
\begin{align*}
  H_\Q^{[\K : \Q]}(q) &= H_\K(q)\\
  &\leq (2T)^{d[\K : \Q]}.
\end{align*}
Da Northcott für $\Q$ gilt, haben wir also nur endlich viele Möglichkeiten $q$ zu wählen, also ebenfalls endlich viele für $P$, da die $q$ die Koeffizienten von $P$ sind. Da $x$ eine Nullstelle von $P$ ist folgt die Behauptung, da es auch für $x$ nur endlich viele Möglichkeiten geben kann.
%$\xRightarrow{abc}$
\end{document}
