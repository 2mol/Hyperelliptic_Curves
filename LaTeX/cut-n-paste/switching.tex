\documentclass[english,11pt,a4paper]{article}
\usepackage{preamble}

\begin{document}

Starting with the old assumption that al $x_i$ are pairwise distinct, we rewrite the Vandermonde matrix with $h = x_2 -x_1$ as
\begin{align*}
  V_h =
  \begin{pmatrix}
  & \dots \\
  1 & x_1 + h & (x_1 + h)^2 & (x_1 + h)^3\\
  & \dots
  \end{pmatrix}
\end{align*}

ugh, another time...



\newpage
--- old stuff below ---

General case: all $x_i$ are pairwise distinct. Define $h = x_2 - x_1$ which is non-zero for the moment. Now write
\begin{align*}
  \begin{pmatrix}
  1 & x_1 & x_1^2 & x_1^3\\
  1 & x_2 & x_2^2 & x_2^3\\
  1 & x_3 & x_3^2 & x_3^3\\
  1 & x_4 & x_4^2 & x_4^3\\
  \end{pmatrix}
  \begin{pmatrix}
  p_0 \\
  p_1 \\
  p_2 \\
  p_3 \\
  \end{pmatrix}
  &=
  \begin{pmatrix}
  y_1 \\
  y_2 \\
  y_3 \\
  y_4 \\
  \end{pmatrix}
  \\
  \leadsto
  \begin{pmatrix}
  1 & x_1 & x_1^2 & x_1^3\\
  1 & x_1+h & (x_1+h)^2 & (x_1+h)^3\\
  1 & x_3 & x_3^2 & x_3^3\\
  1 & x_4 & x_4^2 & x_4^3\\
  \end{pmatrix}
  \begin{pmatrix}
  p_0 \\
  p_1 \\
  p_2 \\
  p_3 \\
  \end{pmatrix}
  &=
  \begin{pmatrix}
  y_1 \\
  y_2 \\
  y_3 \\
  y_4 \\
  \end{pmatrix}
  \\
  \leadsto
  \begin{pmatrix}
  1 & x_1 & x_1^2 & x_1^3\\
  1 & x_1+h & x_1^2 + 2hx_1 + h^2 & x_1^3+3x_1^2h+3x_1h^2+h^3\\
  1 & x_3 & x_3^2 & x_3^3\\
  1 & x_4 & x_4^2 & x_4^3\\
  \end{pmatrix}
  \begin{pmatrix}
  p_0 \\
  p_1 \\
  p_2 \\
  p_3 \\
  \end{pmatrix}
  &=
  \begin{pmatrix}
  y_1 \\
  y_2 \\
  y_3 \\
  y_4 \\
  \end{pmatrix}
  \\
  \leadsto
  \begin{pmatrix}
  1 & x_1 & x_1^2 & x_1^3\\
  0 & h & h(2x_1 + h) & h(3x_1^2+3x_1h+h^2)\\
  1 & x_3 & x_3^2 & x_3^3\\
  1 & x_4 & x_4^2 & x_4^3\\
  \end{pmatrix}
  \begin{pmatrix}
  p_0 \\
  p_1 \\
  p_2 \\
  p_3 \\
  \end{pmatrix}
  &=
  \begin{pmatrix}
  y_1 \\
  y_2-y_1 \\
  y_3 \\
  y_4 \\
  \end{pmatrix}
  \\
  \leadsto
  \begin{pmatrix}
  1\\ 
  & h\\
  & & 1\\
  & & & 1\\
  \end{pmatrix}
  \begin{pmatrix}
  1 & x_1 & x_1^2 & x_1^3\\
  0 & 1 & 2x_1 + h & 3x_1^2+3x_1h+h^2\\
  1 & x_3 & x_3^2 & x_3^3\\
  1 & x_4 & x_4^2 & x_4^3\\
  \end{pmatrix}
  \begin{pmatrix}
  p_0 \\
  p_1 \\
  p_2 \\
  p_3 \\
  \end{pmatrix}
  &=
  \begin{pmatrix}
  1\\ 
  & h\\
  & & 1\\
  & & & 1\\
  \end{pmatrix}
  \begin{pmatrix}
  y_1 \\
  \frac{y_2-y_1}{h} \\
  y_3 \\
  y_4 \\
  \end{pmatrix}
  \\
  \leadsto
  \begin{pmatrix}
  1 & x_1 & x_1^2 & x_1^3\\
  0 & 1 & 2x_1 + h & 3x_1^2+3x_1h+h^2\\
  1 & x_3 & x_3^2 & x_3^3\\
  1 & x_4 & x_4^2 & x_4^3\\
  \end{pmatrix}
  \begin{pmatrix}
  p_0 \\
  p_1 \\
  p_2 \\
  p_3 \\
  \end{pmatrix}
  &=
  \begin{pmatrix}
  y_1 \\
  \frac{y_2-y_1}{h} \\
  y_3 \\
  y_4 \\
  \end{pmatrix}
\end{align*}
As $\lim_{h \to 0} \frac{y_2-y_1}{h} = \frac{dy}{dx} = \frac{C'(x_1)}{2y_1}$ we get the exact same system as in case 2.

-- xxx -- Blah, and so on, this leads to all tangential cases.

For the remaining case $\{ \q_1, \q_2\} + \{\bar \q_1, \q_4 \}$ take a more direct approach. Define $M_h = \{ (x,y) \mid y - p_h(x) = 0 \}$ for a $p_h$ given above. Specifically, under the right conditions of pairwise-distinctness and $p^*(x) = \det V \cdot p(x)$,
\begin{align*}
  p^*(x)=
  \begin{vmatrix}
  1 & x_1 & x_1^2 & y_1\\
  1 & x_2 & x_2^2 & y_2\\
  1 & x_3 & x_3^2 & y_3\\
  1 & x_4 & x_4^2 & y_4\\
  \end{vmatrix}
  x^3 +
  \begin{vmatrix}
  1 & x_1 & y_1 & x_1^3\\
  1 & x_2 & y_2 & x_2^3\\
  1 & x_3 & y_3 & x_3^3\\
  1 & x_4 & y_4 & x_4^3\\
  \end{vmatrix}
  x^2 +
  \begin{vmatrix}
  1 & y_1 & x_1^2 & x_1^3\\
  1 & y_2 & x_2^2 & x_2^3\\
  1 & y_3 & x_3^2 & x_3^3\\
  1 & y_4 & x_4^2 & x_4^3\\
  \end{vmatrix}
  x +
  \begin{vmatrix}
  y_1 & x_1 & x_1^2 & x_1^3\\
  y_2 & x_2 & x_2^2 & x_2^3\\
  y_3 & x_3 & x_3^2 & x_3^3\\
  y_4 & x_4 & x_4^2 & x_4^3\\
  \end{vmatrix}
\end{align*}
so we look at
\begin{align*}
  p_h(x) &=
  \begin{vmatrix}
  1 & x_1 & x_1^2 & y_1\\
  1 & x_1+h & (x_1 + h)^2 & y_2\\
  1 & x_3 & x_3^2 & y_3\\
  1 & x_4 & x_4^2 & y_4\\
  \end{vmatrix}
  x^3 +
  \begin{vmatrix}
  1 & x_1 & y_1 & x_1^3\\
  1 & x_1+h & y_2 & (x_1 + h)^3\\
  1 & x_3 & y_3 & x_3^3\\
  1 & x_4 & y_4 & x_4^3\\
  \end{vmatrix}
  x^2 \\&+
  \begin{vmatrix}
  1 & y_1 & x_1^2 & x_1^3\\
  1 & y_2 & (x_1 + h)^2 & (x_1 + h)^3\\
  1 & y_3 & x_3^2 & x_3^3\\
  1 & y_4 & x_4^2 & x_4^3\\
  \end{vmatrix}
  x +
  \begin{vmatrix}
  y_1 & x_1 & x_1^2 & x_1^3\\
  y_2 & x_1+h & (x_1 + h)^2 & (x_1 + h)^3\\
  y_3 & x_3 & x_3^2 & x_3^3\\
  y_4 & x_4 & x_4^2 & x_4^3\\
  \end{vmatrix}
  \\&=
  % \begin{vmatrix}
  % 1 & x_1 & x_1^2 & y_1\\
  % 1 & x_1+h & (x_1 + h)^2 & y_2\\
  % 1 & x_3 & x_3^2 & y_3\\
  % 1 & x_4 & x_4^2 & y_4\\
  % \end{vmatrix}
\end{align*}
\end{document}